\graphicspath{ {./chapter4/images/} }

\chapter{Реализация, тестване}


\section{Реализация на класове}

Ключови за проекта са класовете \texttt{Satellite} и \texttt{Integrator}. Класът
\texttt{Integrator} е абстрактен и бива използван за реализацията на методите на
Ойлер, Верле и Рунге-Кута. Класът \texttt{Satellite} съдържа описанието на
орбитата си и може да превръща Кеплеровите елементи във вектори на състоянието и
обратно.


\section{Управление на паметта и алгоритми. Оптимизации.}

Местата, на които бе необходимо ръчно управление на паметта, са най-вече
масивите от генерирани данни от симулацията. Те се управляват от класът
\texttt{Integrator}, като се освобождават в неговия деструктор според
принципа RAII.

При симулацията най-важни са алгоритмите за пресмятане на позицията и скоростта
на сателита, както и тези за превръщане на Кеплеровите елементи в Декартови и
обратното \cite{cart_to_kepl} \cite{kepl_to_cart} \cite{classical_orb_elem}. Те
са оптимизирани да работят със стойности без размерност, което
увеличава тяхната точност. Размерността се пренася изцяло от следните три
константи:

\begin{align*}
	R_0 &= R_E \\
	V_0 &= \sqrt{\frac{M_E G}{R_E}} \\
	T_0 &= \frac{R_0}{V_0}
\end{align*}

След нормиране на векторите на позицията и скоростта получаваме следните
диференциални уравнения

\begin{align*}
	\vec{r} &= R_0 \vec{\rho} \Rightarrow \frac{d\vec{r}}{dt} = \cancel{\frac{R_0}{T_0}} \frac{d\vec{\rho}}{d\tau} = \cancel{V_0} \vec{u} \Rightarrow \frac{d\vec{\rho}}{d\tau} = \vec{u} \\
	\vec{v} &= V_0 \vec{u} \Rightarrow \frac{d\vec{v}}{dt} = \cancel{\frac{V_0}{T_0}} \frac{d\vec{u}}{d\tau} = - \cancel{\frac{M_E G}{R_0^3}} \frac{\vec{\rho}}{\left\lvert \rho \right\rvert^3 } \Rightarrow \frac{d\vec{u}}{d\tau} = -\frac{\vec{\rho}}{\left\lvert \rho \right\rvert^3 } \\
	t &= T_0 \tau
\end{align*}


\section{Планиране, описание и създаване на тестови сценарии}

За тестване на кода бе използвана библиотеката GoogleTest. Класът
\texttt{Integrator} и неговите наследници бяха тествани чрез т.н. Fixture,
който позволява преизползване на инициализирани променливи в множество тестове

