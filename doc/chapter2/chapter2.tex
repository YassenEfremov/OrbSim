\graphicspath{ {./chapter2/images/} }

\chapter{Преглед на предметната област}


\section{Основни дефиниции, концепции и алгоритми, които ще бъдат използвани}

Всяка орбита може да бъде описана напълно от 6 стойности. Има два начина на описание:

\begin{itemize}
	\item Чрез 6-те орбитални елемента, задаващи Кеплерова орбита:
	\begin{itemize}
		\item Ексцентрицитет \( e \)
		\item Голяма полуос \( a \)
		\item Наклон на орбитата \( i \) 
		\item Дължината на възходящия връх \( \Omega \)
		\item Параметър на перихелия \( \omega \)
		\item Истинска аномалия \( \nu \)
	\end{itemize}
	\item Чрез вектори на орбиталното състояние, задаващи позицията и скоростта на тялото:
	\begin{itemize}
		\item Вектор на позицията \( \vec{r} = \left( r_x, r_y, r_z \right) \in \mathbb{R}^3 \)
		\item Вектор на скоростта \( \vec{v} = \left( v_x, v_y, v_z \right) \in \mathbb{R}^3 \)
	\end{itemize}
\end{itemize}

Самата орбита представлява конично сечение, като обикаляното тяло се намира в единия от неговите фокуси.

За простота приемаме, че обикаляното тяло е идеална сфера с маса  и гравитационна константа .

Разполагаме с диференциалните уравнения, описващи начина, по който се променя скоростта и позицията на сателита с времето:

\begin{align*}
	\frac{d\vec{v}}{dt} &= -\frac{MG}{|\vec{r}|^2}\frac{\vec{r}}{|\vec{r}|} \\
	\frac{d\vec{r}}{dt} &= \vec{v}
\end{align*}

Да намерим аналитично решение на такива уравнения е прекалено трудно (дори често невъзможно). Можем обаче да намерим тяхно числено решение. Това означава, че разполагаме с начални стойности за търсената функция.
След това избираме дискретна времева стъпка  и с нейна помощ намираме следващата стойност на функцията. Така получаваме апроксимация на търсената функция. Този метод се нарича метод на Ойлер. При него обаче се натрупва грешка. Тя може да бъде минимизирана чрез методите на Верле или на Рунге-Кута.


% \section{Дефиниране на проблеми и сложност на поставената задача}




% \section{Подходи и методи за решаване на поставените проблеми}




\section{Потребителски изисквания и качествени изисквания}

Програмата трябва да предоставя графичен потребителски интерфейс (GUI), чрез
който да могат да се задават начални стойности на симулацията.

