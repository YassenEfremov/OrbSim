\chapter*{Заключение}

Настоящата дипломна работа реализира базов вариант на система за управление на
пакети за програмните езици C и C++. Системата бе написана на C++, като е
достъпна за платформите Linux и Windows. Тя предоставя конзолен потребителски
интерфейс, посредством който могат да бъдат изпълнени команди, позволяващи
инсталирането, премахването, изброяването, създаването и синхронизирането на
софтуерни пакети. Като регистър за съхранение на пакетите бе използвана
платформата GitHub. Системата предоставя механизъм за автоматично управление на
зависимостите на пакетите посредством символични връзки (symlinks). Други
допълнителни функционалности са воденето на диагностични записки и показването
на анимирани индикатори за напредък по време на изпълнението на програмата.

Възможностите за бъдещо развитие на проекта са многобройни: следващата голяма
цел би била разработката на собствен регистър за съхранение на пакети,
предоставящ специално разработени точки за достъп (endpoints), позволяващи
по-лесно сдобиване с кода на пакетите и техните зависимости; други цели са
автоматична компилация на кода при теглене на пакетите, поддръжка за други build
системи освен CMake (например Meson и Bazel) и много допълнителни команди,
улесняващи потребителя - команда за търсене на пакети, команда за актуализация
на вече инсталирани пакети, команда за автоматично публикуване на новосъздадени
пакети и други.
